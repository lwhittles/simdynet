\documentclass[a4paper]{article}\usepackage[]{graphicx}\usepackage[]{color}
%% maxwidth is the original width if it is less than linewidth
%% otherwise use linewidth (to make sure the graphics do not exceed the margin)
\makeatletter
\def\maxwidth{ %
  \ifdim\Gin@nat@width>\linewidth
    \linewidth
  \else
    \Gin@nat@width
  \fi
}
\makeatother

\definecolor{fgcolor}{rgb}{0.345, 0.345, 0.345}
\newcommand{\hlnum}[1]{\textcolor[rgb]{0.686,0.059,0.569}{#1}}%
\newcommand{\hlstr}[1]{\textcolor[rgb]{0.192,0.494,0.8}{#1}}%
\newcommand{\hlcom}[1]{\textcolor[rgb]{0.678,0.584,0.686}{\textit{#1}}}%
\newcommand{\hlopt}[1]{\textcolor[rgb]{0,0,0}{#1}}%
\newcommand{\hlstd}[1]{\textcolor[rgb]{0.345,0.345,0.345}{#1}}%
\newcommand{\hlkwa}[1]{\textcolor[rgb]{0.161,0.373,0.58}{\textbf{#1}}}%
\newcommand{\hlkwb}[1]{\textcolor[rgb]{0.69,0.353,0.396}{#1}}%
\newcommand{\hlkwc}[1]{\textcolor[rgb]{0.333,0.667,0.333}{#1}}%
\newcommand{\hlkwd}[1]{\textcolor[rgb]{0.737,0.353,0.396}{\textbf{#1}}}%
\let\hlipl\hlkwb

\usepackage{framed}
\makeatletter
\newenvironment{kframe}{%
 \def\at@end@of@kframe{}%
 \ifinner\ifhmode%
  \def\at@end@of@kframe{\end{minipage}}%
  \begin{minipage}{\columnwidth}%
 \fi\fi%
 \def\FrameCommand##1{\hskip\@totalleftmargin \hskip-\fboxsep
 \colorbox{shadecolor}{##1}\hskip-\fboxsep
     % There is no \\@totalrightmargin, so:
     \hskip-\linewidth \hskip-\@totalleftmargin \hskip\columnwidth}%
 \MakeFramed {\advance\hsize-\width
   \@totalleftmargin\z@ \linewidth\hsize
   \@setminipage}}%
 {\par\unskip\endMakeFramed%
 \at@end@of@kframe}
\makeatother

\definecolor{shadecolor}{rgb}{.97, .97, .97}
\definecolor{messagecolor}{rgb}{0, 0, 0}
\definecolor{warningcolor}{rgb}{1, 0, 1}
\definecolor{errorcolor}{rgb}{1, 0, 0}
\newenvironment{knitrout}{}{} % an empty environment to be redefined in TeX

\usepackage{alltt}

%\VignetteIndexEntry{UsingSimplasex}
%\VignetteEngine{knitr::knitr}

\usepackage{geometry}
\geometry{tmargin=2.5cm,bmargin=2.5cm,lmargin=2.5cm,rmargin=2.5cm}

\title{Tutorial introducing the R package simplasex}
\author{Lilith Whittles and Xavier Didelot}
\IfFileExists{upquote.sty}{\usepackage{upquote}}{}
\begin{document}


\maketitle

\section{Introduction}

The R package simplasex provides functions to simulate realistic sexual networks.
If you have not already done so, you can install simplasex using the following R command:

\begin{knitrout}
\definecolor{shadecolor}{rgb}{0.969, 0.969, 0.969}\color{fgcolor}\begin{kframe}
\begin{alltt}
\hlstd{devtools}\hlopt{::}\hlkwd{install_github}\hlstd{(}\hlstr{'lwhittles/simplasex'}\hlstd{)}
\end{alltt}
\end{kframe}
\end{knitrout}

You should then be able to load simplasex using:
\begin{knitrout}
\definecolor{shadecolor}{rgb}{0.969, 0.969, 0.969}\color{fgcolor}\begin{kframe}
\begin{alltt}
\hlkwd{library}\hlstd{(}\hlstr{'simplasex'}\hlstd{)}
\end{alltt}
\end{kframe}
\end{knitrout}

Finally, if you want to reproduce exactly the same results as the ones shown in this tutorial, you should set the seed of your random number generator to the same as ours:
\begin{knitrout}
\definecolor{shadecolor}{rgb}{0.969, 0.969, 0.969}\color{fgcolor}\begin{kframe}
\begin{alltt}
\hlkwd{set.seed}\hlstd{(}\hlnum{0}\hlstd{)}
\end{alltt}
\end{kframe}
\end{knitrout}

Throughout this vignette we use a population of size 1000 which is too small to be realistic in most settings, but this is useful to make execution of commands fast for illustration purposes.

\section{Static network simulation}

We can create a static snapshot of a sexual network observed over a year using:

\begin{knitrout}
\definecolor{shadecolor}{rgb}{0.969, 0.969, 0.969}\color{fgcolor}\begin{kframe}
\begin{alltt}
\hlstd{s} \hlkwb{<-} \hlkwd{sim_static_sn}\hlstd{(}\hlkwc{N} \hlstd{=} \hlnum{1000}\hlstd{,} \hlkwc{gamma} \hlstd{=} \hlnum{1.8}\hlstd{)}
\end{alltt}
\end{kframe}
\end{knitrout}

This function can take four parameters, which will be the same for the other network building functions described below:
\begin{itemize}
\setlength{\itemsep}{0pt}
\setlength{\parskip}{0pt}
\item \verb+N+ is the size of the population, the default is \verb+N=1e4+
\item \verb+gamma+ is the exponent of the power law of the degree distribution, the default is \verb+gamma=1.8+
\item \verb+k0+ determines the proportion of individuals without any relationship, the default is \verb+k0=0.5+
\item \verb+phi+ determines the rate at which relationships are evolving, the default is \verb+phi=N+
\end{itemize}

Let's plot the degree distribution of this network:
\begin{knitrout}
\definecolor{shadecolor}{rgb}{0.969, 0.969, 0.969}\color{fgcolor}\begin{kframe}
\begin{alltt}
\hlkwd{plot}\hlstd{(s}\hlopt{$}\hlstd{dd,}\hlkwc{xlab} \hlstd{=} \hlstr{''}\hlstd{,}\hlkwc{ylab} \hlstd{=} \hlstr{'Degree distribution'}\hlstd{,}\hlkwc{log} \hlstd{=} \hlstr{'xy'}\hlstd{)}
\end{alltt}
\end{kframe}

{\centering \includegraphics[width=4in,height=4in]{/var/folders/36/jfh5wwjd5ps05vj02q2b7dk00000gp/T//RtmpIgRdy3unnamed-chunk-6-1} 

}



\end{knitrout}

\section{Simulation of outbreak in static network}

Simulation of outbreaks on static sexual networks is not recommended as it is 
unrealistic of the way disease and network evolve on the same time scale.
This functionality is provided only for comparison with the more realistic dynamic method (cf below).
To simulate an outbreak on the static network generated in the previous
section, we can use:

\begin{knitrout}
\definecolor{shadecolor}{rgb}{0.969, 0.969, 0.969}\color{fgcolor}\begin{kframe}
\begin{alltt}
\hlstd{s} \hlkwb{<-} \hlkwd{sim_outbreak_static_sn}\hlstd{(}\hlkwc{sn} \hlstd{= s)}
\end{alltt}
\end{kframe}
\end{knitrout}

Or alternatively we can simulate both the static network and the outbreak using a single command:

\begin{knitrout}
\definecolor{shadecolor}{rgb}{0.969, 0.969, 0.969}\color{fgcolor}\begin{kframe}
\begin{alltt}
\hlstd{s} \hlkwb{<-} \hlkwd{sim_outbreak_static_sn}\hlstd{(}\hlkwc{N} \hlstd{=} \hlnum{1000}\hlstd{,} \hlkwc{gamma} \hlstd{=} \hlnum{1.8}\hlstd{)}
\end{alltt}
\end{kframe}
\end{knitrout}

Let's plot the degree distribution of this network:
\begin{knitrout}
\definecolor{shadecolor}{rgb}{0.969, 0.969, 0.969}\color{fgcolor}\begin{kframe}
\begin{alltt}
\hlkwd{plot}\hlstd{(s}\hlopt{$}\hlstd{dd,}\hlkwc{xlab} \hlstd{=} \hlstr{''}\hlstd{,}\hlkwc{ylab} \hlstd{=} \hlstr{'Degree distribution'}\hlstd{,}\hlkwc{log} \hlstd{=} \hlstr{'xy'}\hlstd{)}
\end{alltt}
\end{kframe}

{\centering \includegraphics[width=4in,height=4in]{/var/folders/36/jfh5wwjd5ps05vj02q2b7dk00000gp/T//RtmpIgRdy3unnamed-chunk-9-1} 

}



\end{knitrout}

Histogram of infection times:
\begin{knitrout}
\definecolor{shadecolor}{rgb}{0.969, 0.969, 0.969}\color{fgcolor}\begin{kframe}
\begin{alltt}
\hlstd{infs}\hlkwb{=}\hlstd{s}\hlopt{$}\hlstd{log_infs}\hlopt{$}\hlstd{time[}\hlkwd{which}\hlstd{(s}\hlopt{$}\hlstd{log_infs}\hlopt{$}\hlstd{c1}\hlopt{==}\hlstr{'S'}\hlopt{&}\hlstd{s}\hlopt{$}\hlstd{log_infs}\hlopt{$}\hlstd{c2}\hlopt{==}\hlstr{'U'}\hlstd{)]}
\hlkwd{hist}\hlstd{(infs,}\hlkwc{xlab}\hlstd{=}\hlstr{'Time'}\hlstd{,}\hlkwc{ylab}\hlstd{=}\hlstr{'Infections'}\hlstd{,}\hlkwc{breaks} \hlstd{=} \hlnum{20}\hlstd{,} \hlkwc{main}\hlstd{=}\hlstr{''}\hlstd{)}
\end{alltt}
\end{kframe}

{\centering \includegraphics[width=4in,height=4in]{/var/folders/36/jfh5wwjd5ps05vj02q2b7dk00000gp/T//RtmpIgRdy3unnamed-chunk-10-1} 

}



\end{knitrout}

\section{Dynamic network simulation}

We can simulate a dynamic network using:

\begin{knitrout}
\definecolor{shadecolor}{rgb}{0.969, 0.969, 0.969}\color{fgcolor}\begin{kframe}
\begin{alltt}
\hlstd{s} \hlkwb{<-} \hlkwd{sim_dynamic_sn}\hlstd{(}\hlkwc{N} \hlstd{=} \hlnum{1000}\hlstd{,} \hlkwc{gamma} \hlstd{=} \hlnum{1.8}\hlstd{)}
\end{alltt}
\end{kframe}
\end{knitrout}

Let's plot the degree distribution of this network:
\begin{knitrout}
\definecolor{shadecolor}{rgb}{0.969, 0.969, 0.969}\color{fgcolor}\begin{kframe}
\begin{alltt}
\hlkwd{plot}\hlstd{(s}\hlopt{$}\hlstd{dd,}\hlkwc{xlab} \hlstd{=} \hlstr{''}\hlstd{,}\hlkwc{ylab} \hlstd{=} \hlstr{'Degree distribution'}\hlstd{,}\hlkwc{log} \hlstd{=} \hlstr{'xy'}\hlstd{)}
\end{alltt}
\end{kframe}

{\centering \includegraphics[width=4in,height=4in]{/var/folders/36/jfh5wwjd5ps05vj02q2b7dk00000gp/T//RtmpIgRdy3unnamed-chunk-12-1} 

}



\end{knitrout}

\section{Simulation of outbreak in dynamic network}

We can simulate a dynamic network and an outbreak simultaneously using:

\begin{knitrout}
\definecolor{shadecolor}{rgb}{0.969, 0.969, 0.969}\color{fgcolor}\begin{kframe}
\begin{alltt}
\hlstd{s} \hlkwb{<-} \hlkwd{sim_dynamic_sn}\hlstd{(}\hlkwc{N} \hlstd{=} \hlnum{1000}\hlstd{,} \hlkwc{gamma} \hlstd{=} \hlnum{1.8}\hlstd{,} \hlkwc{n_infs} \hlstd{=} \hlnum{1}\hlstd{,}\hlkwc{beta}\hlstd{=}\hlnum{2000}\hlstd{)}
\end{alltt}
\end{kframe}
\end{knitrout}

Let's plot the degree distribution of this network:
\begin{knitrout}
\definecolor{shadecolor}{rgb}{0.969, 0.969, 0.969}\color{fgcolor}\begin{kframe}
\begin{alltt}
\hlkwd{plot}\hlstd{(s}\hlopt{$}\hlstd{dd,}\hlkwc{xlab} \hlstd{=} \hlstr{''}\hlstd{,}\hlkwc{ylab} \hlstd{=} \hlstr{'Degree distribution'}\hlstd{,}\hlkwc{log} \hlstd{=} \hlstr{'xy'}\hlstd{)}
\end{alltt}
\end{kframe}

{\centering \includegraphics[width=4in,height=4in]{/var/folders/36/jfh5wwjd5ps05vj02q2b7dk00000gp/T//RtmpIgRdy3unnamed-chunk-14-1} 

}



\end{knitrout}

Histogram of infection times:
\begin{knitrout}
\definecolor{shadecolor}{rgb}{0.969, 0.969, 0.969}\color{fgcolor}\begin{kframe}
\begin{alltt}
\hlstd{infs}\hlkwb{=}\hlstd{s}\hlopt{$}\hlstd{log_infs}\hlopt{$}\hlstd{time[}\hlkwd{which}\hlstd{(s}\hlopt{$}\hlstd{log_infs}\hlopt{$}\hlstd{c1}\hlopt{==}\hlstr{'S'}\hlopt{&}\hlstd{s}\hlopt{$}\hlstd{log_infs}\hlopt{$}\hlstd{c2}\hlopt{==}\hlstr{'U'}\hlstd{)]}
\hlkwd{hist}\hlstd{(infs,}\hlkwc{xlab}\hlstd{=}\hlstr{'Time'}\hlstd{,}\hlkwc{ylab}\hlstd{=}\hlstr{'Infections'}\hlstd{,}\hlkwc{breaks} \hlstd{=} \hlnum{20}\hlstd{,} \hlkwc{main}\hlstd{=}\hlstr{''}\hlstd{)}
\end{alltt}
\end{kframe}

{\centering \includegraphics[width=4in,height=4in]{/var/folders/36/jfh5wwjd5ps05vj02q2b7dk00000gp/T//RtmpIgRdy3unnamed-chunk-15-1} 

}



\end{knitrout}

\end{document}
